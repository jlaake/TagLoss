% Converted from Microsoft Word to LaTeX
% by Chikrii Softlab Word2TeX converter (version 5.0)
% Copyright (C) 1999-2011 Chikrii Softlab. All rights reserved.
% http://www.chikrii.com
% mailto: support@chikrii.com
% License: CSL#004DE0

\documentclass{article}
\usepackage{latexsym}

\begin{document}
\begin{center}
\textbf{Incorporating Dependence in Tag Loss Estimation}
\end{center}

\textbf{Introduction}

Developing methods of marking animals that minimize or eliminate mark loss 
are important because mark-recapture estimators assume recaptures of marked 
individuals are always detected (Seber 1982). However, current methods of 
marking many different species indicate that problems with mark retention 
encompass small mammals (Fokidis et al. 2006), large terrestrial mammals 
(Fosgate et al. 2006), aquatic mammals (Bradshaw et al. 2000), fish (Cowen 
and Schwartz 2006), and reptiles (Rivalan et al. 2005). If the assumption 
that no marks are lost is violated then estimates of population parameters 
(e.g., survival and abundance) will be biased (Arnason and Mills 1981, 
Diefenbach and Alt 1998) because previously marked animals that lose all 
marks are treated as being part of the unmarked population in closed 
population models and as immigrants in open models. Furthermore, even if an 
estimator is robust to violation of this assumption, the loss of marks will 
reduce the precision of parameter estimates because of the apparent 
reduction in sample size (Rotella and Hines 2005).

Using natural markings to track individual animals can have low error rates 
(Stevick et al. 2001), but not all species can be monitored in this manner. 
Consequently, methods of estimating tag loss likely will be necessary until 
new technologies are developed that eliminate mark loss. Historically, 
animals have been double-marked and the status of marks upon recapture (none 
or one mark missing) have been used to estimate mark loss under the 
assumption that each mark is lost independently of the other mark (Beverton 
and Holt 1957, Seber 1982). This independence assumption is required because 
the recapture of a marked animal that lost both marks is not observable.

In recent studies where the opportunity has occurred to observe loss of both 
tags via the use of a third permanent mark, however, the independence 
assumption has been shown to be invalid (Siniff and Ralls, 1991; Diefenbach 
and Alt, 1998; Bradshaw, Barker and Davis, 2000; Rivalan et al., 2005). 
These studies undermine the credibility of tag loss evaluations for 
situations in which loss of both marks is not observable (Pistorius et al., 
2000). Consequently, it is important to incorporate appropriate models of 
mark loss to reduce bias in estimates of population parameters, even if 
modeling tag loss results in reduced precision (e.g., Pollock et al. 2007).

Unfortunately, not all species can be marked with a third permanent mark to 
detect the recapture of individuals with both marks missing to assess the 
assumption of independence of each mark loss event. This situation has some 
similarities to mark-recapture distance sampling (Laake, 1999; Borchers et 
al., 2006; Laake, Dawson and Hone, 2008) in which objects missed by both 
observers cannot be included in the sample because they are not observed 
(Borchers, 1999). However, \textless \textless appropriate Borchers/Laake 
citation here\textgreater \textgreater used ancillary distance data to 
weaken the independence assumption such that independence was assumed only 
for observations that occurred close to the transect line and improve 
abundance estimates. If the independence assumption can be similarly 
weakened in situations where both marks are lost in mark-recaptures studies, 
it may be possible to evaluate loss of marks for species where only 
double-marked individuals are feasible.

We develop a model of mark loss that explicitly models dependence in loss of 
marks and then we consider situations where we can weaken the independence 
assumption by incorporating dependence. We apply the model to tagged 
California sea lions (\textit{Zalophus californianus}) and black bear (\textit{Ursus americanus}) which were both double tagged and 
given a separate permanent mark. We compare the results using an 
independence model and dependence model with the double tag data excluding 
the double tag loss observations to the results using the data including the 
known double tag loss.

\textbf{Tag Loss Models}

We first consider a simple situation which has been well described in the 
literature (Seber 1982) and then we will build from this foundation. Two 
tags are applied to a sample of animals, they are released and at some time 
later a sample of animals are observed having one (n$_{\mathrm{1}}=$ 
n$_{\mathrm{01}}+$ n$_{\mathrm{10}})$ or two (n$_{\mathrm{2}}=$ 
n$_{\mathrm{11}})$ tags present. In this simple example, double tag loss 
($n_{00})$ is not observable. Let \textbf{S} $=$ ($S_{\mathrm{1}}$, 
$S_{\mathrm{2}})$ represent the status vector of the 2 tags where $S_{i}$ is 1 
if the $i^{th}$ tag is present and 0 if the $i^{th}$ tag is absent (lost) 
(note: this is counter to the convention of Diefenbach et al. (1998), 
Bradshaw et al. (2000) and Rivalan et al. (2005) who use 1 to denote a tag 
loss event). Usually the tagged animals can only be identified if one or 
more tags are retained which means those that lost both tags 
($n_{00};\thinspace $\textbf{S} $=$ (0,0)) are not observable. Initially, we 
assume both tags have the same probability of tag loss, $p$, and we assume 
independence of tag fates. 

Table 1. Joint ($\pi_{s_{1}s_{2}})$ and marginal ($p$, $q)$ probabilities for 
double tag status \textbf{S} $=$ ($S_{\mathrm{1}}$, $S_{\mathrm{2}})$ with an 
assumed independence structure. The $n_{00}$ cell is shaded gray because it 
is not observed. 

\begin{table}[htbp]
\begin{center}
\begin{tabular}{|l|p{112pt}|p{130pt}|p{135pt}|}
\hline
& 
\multicolumn{2}{|p{243pt}|}{\textbf{Tag 2 Status (S}$_{\mathrm{\mathbf{2}}}$\textbf{)}} & 
 \\
\hline
\textbf{Tag 1 Status (S}$_{\mathrm{\mathbf{1}}}$\textbf{)}& 
\textit{Present (S}$_{2}=$\textit{1)}& 
\textit{Absent (S}$_{2}=$\textit{0)}& 
\textit{Marginal} \\
\hline
\textit{Present (S}$_{1}=$\textit{1)}& 
${n_{11};\thinspace \pi }_{11}={(1-p)}^{2}$& 
${n_{10};\thinspace \pi }_{10}=(1-p)p$& 
$q=1-p$ \\
\hline
\textit{Absent (S}$_{1}=$\textit{0)}& 
${n_{01};\thinspace \pi }_{01}=p(1-p)$& 
$n_{00};\thinspace {\thinspace \pi }_{00}=p^{2}$& 
$p$ \\
\hline
\textit{Marginal}& 
$q=1-p$& 
$p$& 
$1$ \\
\hline
\end{tabular}
\label{tab1}
\end{center}
\end{table}

Based on the independence assumption, the probability that an animal retains 
both tags is (1-p)$^{\mathrm{2}}$, the probability of retaining only one tag 
is 2p(1-p) and the probability of losing both tags is p$^{\mathrm{2}}$. The 
data and probability structure can be represented by a 2x2 contingency table 
for the status (present (retained)/absent (lost)) for each tag (Table 1). 
The marginal probabilities of tag loss and tag retention are $p$ and 
$q=$1-$p$ and the joint probabilities are represented as $\pi_{11}\pi_{10}\pi 
_{01}\pi_{00}\thinspace $for each of the cells in the table. Due to the 
assumed independence the conditional probability of tag loss or retention of 
a tag given the status of the other tag is the same as the marginal 
probability. For example, the probability that tag 1 is lost given that tag 
2 is retained is${\thinspace \pi }_{01}/(1-p)=p\thinspace $

The log-likelihood function (excluding the constant) for $p$ given the observed 
data ($n_{\mathrm{1}}$, $n_{\mathrm{2}})$ is conditioned on the observations 
which excludes $n_{00}:$
\[
L\left( p\thinspace\vert\thinspace {n_{1},n_{2}}\right) \propto n_{1}\log 
_{e}[2{p(1-p)}]+n_{2}\log_{e}{[\left( 1-p 
\right)^{2}]}-(n_{1}+n_{2})\thinspace {log}_{e}{[1-p^{2}]}
\]
The maximum likelihood estimator (MLE) for tag loss probability is 
$\hat{p}=\frac{n_{1}}{n_{1}{+2n}_{2}}$. Now if we allow different 
probabilities for each tag $(p_{1}p_{2})$ (e.g., different tag types or tag 
orientation), then the log-likelihood function is:
\[
L\left( {p_{1},p_{2}}\thinspace\vert\thinspace {n_{10},n_{01},n_{11}}\right) 
\propto n_{10}\log_{e}[(1-p_{1})p_{2}]+n_{01}{\mathrm{log}}_{e}\left[ 
p_{1}\left( 1-p_{2} \right) \right]+n_{11}\log_{e}\left[ \left( 1-p_{1} 
\right)\left( 1-p_{2} \right) \right]-
\]
\[
(n_{01}+n_{10}+n_{11})\thinspace {log}_{e}{[1-p_{1}p_{2}]}
\]
and the MLEs are $\hat{p}_{1}=\frac{n_{01}}{n_{01}{+n}_{11}}$ and 
$\hat{p}_{2}=\frac{n_{10}}{n_{10}{+n}_{11}}$ which are equivalent to the 
capture probability estimators in a two occasion capture-recapture 
experiment for a closed population (i.e., Lincoln-Petersen). Each of those 
estimators can be viewed as a conditional probability because they measure 
the probability that tag $i$ was lost given that tag 3-$i$ was retained. If 
independence holds, the conditional and marginal probabilities are the same. 

We can also express the tag loss probabilities using a logit link (Bradshaw 
et al. 2000) which becomes helpful to incorporate covariates and to express 
dependence in tag fates. With the logit link, the natural logarithm 
(log$_{\mathrm{e}})$ of the odds (log$_{\mathrm{e}}$(p/(1-p)) is some linear 
function of the parameters. Odds are simply the ratio of the probabilities 
of the event occurring and not occurring. Allowing different tag loss 
probabilities for each tag, we can re-express the probabilities in Table 1 
such that the odds of losing the $i^{th}$ tag ($S_{i}=$0) is $e^{\beta 
_{i}}$ (Table 2). The model has the same structure and there is a 1-1 
relationship between $\hat{p}_{i}$ and the parameters $\hat{\beta }_{i}=\log 
_{e}{[\hat{p}_{i}} \mathord{\left/ {\vphantom {{[\hat{p}_{i}} 
{{(1-\hat{p}}_{i})]}}} \right. \kern-\nulldelimiterspace} 
{{(1-\hat{p}}_{i})]}$

Table 2. Joint ($\pi_{s_{1}s_{2}}^{\ast })$ and marginal ($p_{i}^{\ast 
}q_{i}^{\ast })$ probabilities for double tag status \textbf{S} $=$ 
($S_{\mathrm{1}}$, $S_{\mathrm{2}})$, where the odds ($p_{i}^{\ast 
}/q_{i}^{\ast })\thinspace $of losing the $i^{th}$ tag is $e^{\beta_{i}}$ 
and $K^{\ast }=1+e^{\beta_{1}}+e^{\beta_{2}}+e^{{\beta_{1}+\beta 
}_{2}}=(1+e^{\beta_{1}})(1+e^{\beta_{2}})$ The $n_{00}$ cell is shaded gray 
because it is not observed. 

\begin{table}[htbp]
\begin{center}
\begin{tabular}{|l|p{103pt}|p{103pt}|p{171pt}|}
\hline
& 
\multicolumn{2}{|p{207pt}|}{\textbf{Tag 2 Status (S}$_{\mathrm{\mathbf{2}}}$\textbf{)}} & 
 \\
\hline
\textbf{Tag 1 Status (S}$_{\mathrm{\mathbf{1}}}$\textbf{)}& 
\textit{Present (S}$_{2}=$\textit{1)}& 
\textit{Absent (S}$_{2}=$\textit{0)}& 
\textit{Marginal} \\
\hline
\textit{Present (S}$_{1}=$\textit{1)}& 
$\pi_{11}^{\ast }=1/K^{\ast }$& 
$\pi_{10}^{\ast }=e^{\beta_{2}}/K^{\ast }$& 
$q_{1}^{\ast }=\frac{1+e^{\beta_{2}}}{K^{\ast }}=\frac{1}{1+e^{\beta_{1}}}$ \\
\hline
\textit{Absent (S}$_{1}=$\textit{0)}& 
$\pi_{01}^{\ast }=e^{\beta_{1}}/K^{\ast }$& 
${\pi_{00}^{\ast }=e}^{\beta_{1}+\beta_{2}}/K^{\ast }$& 
$p_{1}^{\ast }=\frac{e^{\beta_{1}}\left( 1+e^{\beta_{2}} \right)}{K^{\ast }}=\frac{e^{\beta_{1}}}{1+e^{\beta_{1}}}$ \\
\hline
\textit{Marginal}& 
$q_{2}^{\ast }=1/(1+e^{\beta_{2}})$& 
${{p_{2}^{\ast }=e}^{\beta_{2}}/(1+e}^{\beta_{2}})$& 
$1$ \\
\hline
\end{tabular}
\label{tab2}
\end{center}
\end{table}

Now we expand the set of observed data to include those animals that lost 
both tags ($n_{00})$ and we allow for possible dependence in the fates of 
the tags. In particular, we specify different odds of losing a tag that are 
dependent on the status of the other tag (Table 3). The odds of losing the 
$i^{th}$ tag ($S_{i}=$0) given the other tag is present ($S_{3-i\thinspace 
}=$1) is $e^{\beta_{i}}$ and the odds of losing the $i^{th}$ tag 
($S_{i}=$0) given other tag is absent ($S_{3-i}=$0) is $e^{\beta 
_{i}+\alpha }$ where $\alpha $ determines the amount of dependence. These 
dependent odds are ratios of the joint probabilities (e.g., ${\pi 
_{01\thinspace }/\pi_{11\thinspace }\thinspace and\thinspace \pi }_{00}/\pi 
_{10})$ and not the marginal probabilities. Also, because of the dependence 
the conditional and marginal probabilities differ.

Table 3. Joint ($\pi_{s_{1}s_{2}})$ and marginal ($p_{i}$, $q_{i})$ 
probabilities for double tag status \textbf{S} $=$ ($S_{\mathrm{1}}$, 
$S_{\mathrm{2}})$, where the odds of losing the $i^{th}$ tag ($S_{i}=$0) 
given the other tag is present ($S_{3-i\thinspace }=$1) is $e^{\beta 
_{i}}$, odds of losing the $i^{th}$ tag ($S_{i}=$0) given other tag is 
absent ($S_{3-i}=$0) is $e^{\beta_{i}+\alpha }$, and $K=1+e^{\beta 
_{1}}+e^{\beta_{2}}+e^{{\beta_{1}+\beta }_{2}+\alpha }$ The $n_{00}$ cell 
is observed in this model. 

\begin{table}[htbp]
\begin{center}
\begin{tabular}{|l|p{112pt}|p{130pt}|p{135pt}|}
\hline
& 
\multicolumn{2}{|p{243pt}|}{\textbf{Tag 2 Status (S}$_{\mathrm{\mathbf{2}}}$\textbf{)}} & 
 \\
\hline
\textbf{Tag 1 Status (S}$_{\mathrm{\mathbf{1}}}$\textbf{)}& 
\textit{Present (S}$_{2}=$\textit{1)}& 
\textit{Absent (S}$_{2}=$\textit{0)}& 
\textit{Marginal} \\
\hline
\textit{Present (S}$_{1}=$\textit{1)}& 
$\pi_{11}=1/K$& 
$\pi_{10}=e^{\beta_{2}}/K$& 
$q_{1}=(1+e^{\beta_{2}})/K$ \\
\hline
\textit{Absent (S}$_{1}=$\textit{0)}& 
$\pi_{01}=e^{\beta_{1}}/K$& 
${\pi_{00}=e}^{\beta_{1}+\beta_{2}+\alpha }/K$& 
${{p_{1}=e}^{\beta_{1}}(1+e}^{\beta_{2}+\alpha })/K$ \\
\hline
\textit{Marginal}& 
$q_{2}=(1+e^{\beta_{1}})/K$& 
${{p_{2}=e}^{\beta_{2}}(1+e}^{\beta_{1}+\alpha })/K$& 
$1$ \\
\hline
\end{tabular}
\label{tab3}
\end{center}
\end{table}

We define $p_{s_{3-i}}^{i}$to be the conditional probability that the 
$i^{th}$ tag is lost given the other tag status is $s_{3-i}$:
\[
p_{i}^{s_{3-i}}=Pr\left( S_{i}=0\vert S_{3-i}=s_{3-i} \right)=\frac{e^{\beta 
_{i}+\alpha {\bullet (1-s}_{3-i})}}{1+e^{\beta_{i}+\alpha {\bullet 
(1-s}_{3-i})}}\thinspace \thinspace .
\]
Thus, $p_{i}^{0}=Pr\left( S_{i}=0\vert S_{3-i}=0 \right)=\frac{e^{\beta 
_{i}+\alpha }}{1+e^{\beta_{i}+\alpha }}$ whereas, $p_{i}^{1}=Pr\left( 
S_{i}=0\vert S_{3-i}=1 \right)=\frac{e^{\beta_{i}}}{1+e^{\beta_{i}}}$. The 
conditional probability of tag retention is $q_{i}^{0}=1-p_{i}^{0}$ 
and$\thinspace q_{i}^{1}=1-p_{i}^{1}$ Each joint probability can be 
specified by the product of a marginal and a conditional probability: $\pi 
_{11}=q_{1}^{1}q_{2}=q_{2}^{1}q_{1},\thinspace \thinspace \pi 
_{10}=q_{1}(1-q_{2}^{1})$ and $\thinspace \pi_{01}=q_{2}(1-q_{1}^{1}).$

Diefenbach et al (1998) and Rivalan et al. (2005) specified the joint 
probabilities in Table 3 in a slightly different manner with $\pi_{01}=\pi 
_{10}=p(1-p^{\ast })$, $\pi_{00}=pp^{\ast }$ and $\pi_{11}=1-2p+pp^{\ast 
}$, where p is the marginal tag loss probability and $p*$ is the conditional 
probability of losing a tag given the other tag was absent. Although not 
stated, their conditional probability of losing a tag given the presence of 
the other tag would be ${p(1-p^{\ast })} \mathord{\left/ {\vphantom 
{{p(1-p^{\ast })} {(1-p)}}} \right. \kern-\nulldelimiterspace} {(1-p)}$. In 
both papers the authors used separate functional forms for $p$ and $p*$ and 
specified the independence model by using the same model for both 
($p=p*)$. Their approach is viable but we believe it is preferable to have a 
model with a parameter that controls dependence and with independence 
specified simply as $\alpha =$0. As we show later this enables flexibility 
in modeling and incorporating dependence. Our model was motivated by 
Bradshaw et al. (2000) which used logistic regression and specified 
dependence in tag loss via interactions and measured effects in terms of 
odds multipliers. We note that they also provide a closed form estimator for 
the model in Table 3 with no covariates and $\alpha =$0.

As mentioned previously the probability structure for tag loss is quite 
similar to capture-recapture (mark-recapture) for two occasions with a 
closed population which has been used with two observers to measure 
detection probability in visual surveys. When detection probability is 
measured solely with the mark-recapture data, it is necessary to assume 
independence between the detections by the two observers because those 
missed by both observers ($n_{00})$ are obviously not included in the sample 
(Borchers, 1999). Recently, the independence assumption was weakened 
(Borchers et al., 2006; Laake, 1999; Laake, Dawson and Hone, 2008)(Laake and 
Borchers 2004) in the combined mark-recapture and distance sampling by 
including a dependence measure $\delta (x)$ which was estimated as the 
discrepancy between the detection probability at distance $x$ measured by the 
mark-recapture (double observer) data (based on independence) and the 
distance sampling data. If $\delta (x)=$1 then independence at all 
distances is achieved. Because detection probability at $x=$0 cannot be 
measured from the distance sampling data, the independence assumption for 
the mark-recapture data was required for $x=$0 ($\delta $(\textit{0})$=$1) but not for 
the other distances. 

The dependence structure we have defined for tag loss can be expressed in 
terms of the $\delta $ dependence of Borchers et al. (2006). Under the 
independence model (Table 2 excluding $n_{00})$, the probability that an 
animal would retain at least one tag is:
\[
1-\pi_{00}^{\ast }=\frac{1+e^{\beta_{1}}+e^{\beta_{2}}}{1+e^{\beta 
_{1}}+e^{\beta_{2}}+e^{\beta_{1}+\beta_{2}}}
\]
Likewise for the dependence model (Table 3):
\[
1-\pi_{00}=\frac{1+e^{\beta_{1}}+e^{\beta_{2}}}{1+e^{\beta_{1}}+e^{\beta 
_{2}}+e^{\beta_{1}+\beta_{2}+\alpha }}
\]
The dependence of Borchers et al. (2006) is a ratio that measures the 
distortion between the joint probabilities from the independence model 
(Table 2) and the dependence model (Table 3) which can be expressed as:
\[
\delta =\frac{1-\pi_{00}^{\ast }}{1-\pi_{00}}=\frac{K}{K^{\ast 
}}=1+\frac{e^{\beta_{1}+\beta_{2}}(e^{\alpha }-1)}{1+e^{\beta_{1}+\beta 
_{2}}+e^{\beta_{1}}+e^{\beta_{2}}}=1+(e^{\alpha }-1)\pi_{00}^{\ast }
\]
The same relationship is obtained from $\delta =q_{i}^{1} \mathord{\left/ 
{\vphantom {q_{i}^{1} q_{i}}} \right. \kern-\nulldelimiterspace} 
q_{i}=\frac{K}{K^{\ast }}\thinspace $and from the ratio of any of the joint 
probabilities other than for the (0,0) event which is not used in the 
independence model. The dependence measure can also be expressed in terms of 
covariance (Borchers 1999) which in this case can be expressed as:
\[
\delta 
=1+\frac{\mathrm{cov}(S_{1},S_{2})}{q_{1}q_{2}}=1+\frac{\frac{1}{K}-q_{1}q_{2}}{q_{1}q_{2}}=\frac{\frac{1}{K}}{\frac{K^{\ast 
}}{K^{2}}}=1+(e^{\alpha }-1)\pi_{00}^{\ast }
\]
In general there will likely be positive dependence in tag loss which means 
$\alpha $\textgreater 0 and $\delta $\textgreater 1 but negative dependence 
($\alpha $\textless 0) is possible with a lower bound of $\delta 
$\textgreater 1-$\pi_{00}^{\ast }$

The joint probabilities can be rewritten in terms of $\delta $ as: $\pi 
_{11}=\delta q_{1}q_{2},\thinspace \thinspace \pi_{10}=q_{1}(1-\delta 
q_{2})$ and $\thinspace \pi_{01}=q_{2}\left( 1-\delta q_{1} \right)$ or as 
$\pi_{11}=q_{1}^{1}q_{2}^{1}/\delta ,\thinspace \thinspace \pi 
_{10}=q_{1}^{1}p_{2}^{1}/\delta $ and $\pi_{01}=q_{2}^{1}p_{1}^{1}/\delta 
$. The latter form makes it obvious that once you exclude $n_{00}$ and 
condition on the observed data, the $\delta $ will cancel from the rescaled 
joint probabilities which will only be functions of the $q_{i}^{1}$. This is 
also obvious by examining Table 3 and noting that the joint probabilities 
for the observed set of data ($n_{11}n_{10}n_{01})$ would only be functions 
of $\beta_{i}$ after conditioning on the exclusion of $n_{00}$. The same 
result was shown by Borchers et al. (2006) for mark-recapture distance 
sampling (mrds) but in that case $\delta $ could be estimated from the 
observed distances. For tag loss estimation, there is no equivalent 
ancillary data; however, there are various ways that dependence can be 
incorporated to improve tag loss estimation. 

The most obvious and best way to incorporate dependence is to mark a subset 
of the animals with a permanent mark so for a subset of the data $n_{00}$ is 
observable. For example, all captured black bears in Pennsylvania were given 
2 ear tags but some of the bears were also given a permanent unique lip 
tattoo (Diefenbach et al. 1998) which could be used to identify bears that 
had lost both ear tags. The likelihood for the tagged only 
($n_{11}n_{10}n_{01})$ would be:
\[
L\left( {\beta_{1},\beta_{2}}\thinspace\vert\thinspace 
{n_{10},n_{01},n_{11}}\right) \propto \prod\limits_{\omega =11,01,10} \left( 
\frac{\pi_{\omega }}{1-\pi_{00}} \right)^{n_{\omega }} 
\]
and for the tagged and marked: ($n_{11}^{m}n_{01}^{m}n_{10}^{m}n_{00}^{m})$: 
\[
L\left( {\beta_{1},\beta_{2},\alpha }\thinspace\vert\thinspace 
{n_{11}^{m},n_{10}^{m},n_{01}^{m},n_{00}^{m}}\right) \propto 
\prod\limits_{\omega =11,01,10,00} \pi_{\omega }^{n_{\omega }^{m}} 
\]
The combined likelihood for the data could improve on the precision of the 
tag loss estimates while adjusting for dependence measured from the 
permanently marked bears. In the absence of data on permanently marked 
animals, a Bayesian alternative could be used in which a prior distribution 
for $\alpha $ or $\delta $ could be used with the likelihood for the tagged 
only bears. Preferably the prior would be data-derived from a similar 
species and situation. 

Next we consider incorporating dependence for situations in which all of the 
animals are double-tagged and we cannot observe animals that lose both tags. 
On the surface this appears to be impossible because there is no ancillary 
data as with mrds. However, we can replace the role of ancillary data with 
models that restrict the dependence and still provide estimable parameters. 
For example, consider California sea lions which have a tag applied to both 
fore-flippers as pups. Initially, tag loss may be due to manufacturing or 
application defects which may be independent between tags; however, as the 
animal grows, the expansion of the fore-flipper may put pressure on the tag 
causing the tag to split or may cause tissue damage that allows the tag to 
fall out. Because growth is symmetric, if one tag is lost the other is also 
more likely to be lost. Thus, we could propose a model in which $\alpha 
=$0 for the first year but not subsequent years. As long as tag loss 
parameters $\beta_{i}$ are age-invariant then any dependence beyond the 
first year can be estimated. Conceptually, this is similar to having a 
subset of animals with a permanent mark but instead we have a subset of 
animals for which it is reasonable to assume independence holds. 

\textbf{Literature Cited}

Borchers, D. L., Laake, J. L., Southwell, C., and Paxton, C. G. M. (2006). 
Accommodating unmodeled heterogeneity in double-observer distance sampling 
surveys. \textit{Biometrics} \textbf{62}, 372-378.

Laake, J. (1999). Distance sampling with independent observers: Reducing 
bias from heterogeneity by weakening the conditional independence 
assumption. In \textit{Marine Mammal Survey and Assessment Methods}, 137-148. Rotterdam: A A BALKEMA.

Laake, J., Dawson, M. J., and Hone, J. (2008). Visibility bias in aerial 
survey: mark-recapture, line-transect or both? \textit{Wildlife Research} \textbf{35}, 299-309.

Borchers, D. L. (1999). Composite mark-recapture line transect surveys. In 
\textit{Marine Mammal Survey and Assessment Methods}, 115-126. Rotterdam: A A BALKEMA.

Borchers, D. L., Laake, J. L., Southwell, C., and Paxton, C. G. M. (2006). 
Accommodating unmodeled heterogeneity in double-observer distance sampling 
surveys. \textit{Biometrics} \textbf{62}, 372-378.

Bradshaw, C. J. A., Barker, R. J., and Davis, L. S. (2000). Modeling tag 
loss in New Zealand fur seal pups. \textit{Journal of Agricultural Biological and Environmental Statistics} \textbf{5}, 475-485.

Diefenbach, D. R., and Alt, G. L. (1998). Modeling and evaluation of ear tag 
loss in black bears. \textit{Journal of Wildlife Management} \textbf{62}, 1292-1300.

Pistorius, P. A., Bester, M. N., Kirkman, S. P., and Boveng, P. L. (2000). 
Evaluation of age- and sex-dependent rates of tag loss in southern elephant 
seals. \textit{Journal of Wildlife Management} \textbf{64}, 373-380.

Rivalan, P., Godfrey, M. H., Prevot-Julliard, A. C., and Girondot, M. 
(2005). Maximum likelihood estimates of tag loss in leatherback sea turtles. 
\textit{Journal of Wildlife Management} \textbf{69}, 540-548.

Siniff, D. B., and Ralls, K. (1991). Reproduction, Survival and Tag Loss in 
California Sea Otters. \textit{Marine Mammal Science} \textbf{7}, 211-229.

Alt, G. L., C. R. McClaughlin, and K. H. Pollock. 1985. Ear tag loss by 
black bears in Pennsylvania. Journal of Wildlife Management 49:316-320.

Arnason, A. N., and K. H. Mills. 1981. Bias and loss of precision due to tag 
loss in Jolly-Seber estimates for mark-recapture experiments. Canadian 
Journal of Fisheries and Aquatic Sciences 38:1077-1095.

Bradshaw, C. J. A., Barker, R. J., and Davis, L. S. (2000). Modeling tag 
loss in New Zealand fur seal pups. Journal of Agricultural, Biological, and 
Environmental Statistics 5:475--485.

Cowen, L., and C. J. Schwartz. 2006. The Jolly-Seber model with tag loss. 
Biometrics 62:699-705.

Fabrizio MC, Nichols JD, Hines JE, Swanson BL, Schram ST. 1999. Modeling 
data from double-tagging experiments to estimate heterogeneous rates of tag 
shedding in lake trout (\textit{Salvelinus namaycush}). Canadian Journal of Fisheries and Aquatic 
Sciences 56:1409-1419.

Fosgate GT, Adesiyun AA, Hird DW. 2006. Ear-tag retention and identification 
methods for extensively managed water buffalo (\textit{Bubalus bubalis}) in Trinidad. Preventive 
Veterinary Medicine 73:287-296.

Pollock KH (Pollock, Kenneth H.), Yoshizaki J (Yoshizaki, Jun), Fabrizio MC 
(Fabrizio, Mary C.), Schram ST (Schram, Stephen T.). 2007. Factors affecting 
survival rates of a recovering lake trout population estimated by 
mark-recapture in Lake Superior, 1969-1996. Transactions of the American 
Fisheries Society 136:185-194.

Rotella, J. J., and J. E. Hines. 2005. Effects of tag loss on direct 
estimates of population growth rate. Ecology 86:821-827.

Stevick PT, Palsboll PJ, Smith TD, Bravington MV, Hammond PS. 2001. Errors 
in identification using natural markings: rates, sources, and effects on 
capture-recapture estimates of abundance. Canadian Journal of Fisheries and 
Aquatic Sciences 58:1861-1870.

\end{document}
